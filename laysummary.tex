%% The following is a directive for TeXShop to indicate the main file
%%!TEX root = diss.tex

%% https://www.grad.ubc.ca/current-students/dissertation-thesis-preparation/preliminary-pages
%% 
%% LAY SUMMARY Effective May 2017, all theses and dissertations must
%% include a lay summary.  The lay or public summary explains the key
%% goals and contributions of the research/scholarly work in terms that
%% can be understood by the general public. It must not exceed 150
%% words in length.

\chapter{Lay Summary}

Defects are tiny imperfections in a crystal where the regular atomic arrangement is disturbed. Far from being flaws, they play a central role in determining a material’s mechanical, thermal, and electrical properties. Different defects give rise to different “flavors” of these properties, and mapping these relationships is key to engineering materials — for example, by precisely controlling dopant defects in semiconductors, we can manipulate electron flow and create devices like transistors, which made the digital era possible. We study these defects using scanning tunneling microscopy (STM), which reveals both atomic positions and electronic behavior. Yet disentangling defect-specific information is challenging because their effects often overlap, much like instruments blending together in an orchestra and obscuring their individual contributions. This thesis takes a step toward resolving that challenge by introducing a statistical model to estimate the population of each defect type and developing a new algorithm to separate their overlapping electronic “fingerprints,” enabling features to be revealed that were previously indistinguishable.
