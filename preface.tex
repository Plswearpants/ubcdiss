%% The following is a directive for TeXShop to indicate the main file
%%!TEX root = diss.tex

\chapter{Preface}
The first part of this thesis presenting original work, presented in Chapter 3, investigates the high mobility of PtSn\textsubscript{4}. High-quality single crystals of PtSn\textsubscript{4} were grown and characterized via transport measurements by Samikshya Sahu (PhD candidate, QMI–UBC). The scanning tunneling microscopy (STM) study of PtSn\textsubscript{4} is original work led by me. Data acquisition was performed by myself, Ashley Warner (MSc, QMI, graduated), Markus Altthaler (Postdoctoral Fellow, QMI), and Seokhwan Choi (Postdoctoral Fellow, QMI, now departed). All work in defect analysis and statistical characterization was completed by me. This research was carried out at the University of British Columbia in the Laboratory for Atomic Imaging Research (LAIR) under the supervision of Prof. Sarah Burke and Prof. Douglas Bonn. Results in Chapter 3 form part of a manuscript currently submitted for publication.

The second part of this thesis, Chapters 5–7, introduces and applies the newly developed \ac{MC-SBD} algorithm, which is publicly available at  \href{https://github.com/Plswearpants/MT-SBD-STM}{this repository} on github. The theoretical multi-channel framework and formulation of the demixing problem presented in Chapter 5 are original contributions by me. The final \ac{MC-SBD} procedure described at the end of Chapter 5 represents my original formulation of the algorithm, based on an earlier version of the algorithmic architecture co-developed with Markus Altthaler. All coding related to the development of the algorithm was performed by me, with assistance from generative AI tools such as Cursor and ChatGPT. A detailed statement on the role of GenAI is provided later.

All work in Chapter 6 is original and was completed by me. Chapter 7 applies the algorithm to four STM grid-map datasets collected on different materials. I took the PtSn\textsubscript{4} grid maps, while previous members of the LAIR group originally measured the other three datasets: Both the Ag(111) grid map in Figure \ref{fig:Ag1} and LiFeAs grid map in Figure \ref{fig:LiFeAs} by Stephanie Grothe, Shun Chi and Yan Pennec. The ZrSiTe grid map in Figure \ref{fig:ZrSiTe1} by Brandon Stuart and Seokhwan Choi. All datasets were preprocessed using a pipeline jointly developed by Brandon Stuart, Seokhwan Choi, and myself. The subsequent algorithm application and data analysis were conducted entirely by me.

Chapters 5–7 present work intended for publication. Chapter 8 is an original, independent, and unpublished discussion written solely by the author.
 

\subsubsection{Use of Generative AI}
The first draft of this thesis was written entirely by me. Grammarly was later used to polish the language and correct grammar mistakes. 

During the development of the \ac{MC-SBD} algorithm, I used generative AI tools including ChatGPT and Cursor to assist in several ways:
\begin{itemize}
	\item Literature Study: I used these tools to help me understand key concepts from the relevant literature and clarify technical points while studying prior work.
	\item Algorithm Development: After the architecture of the algorithm was co-designed by Markus Altthaler and me, I prototyped the core functions. During the iteration phase, I used Cursor as an interactive companion to write test cases, explore alternative implementations, and refactor code for improved readability and maintainability.
	\item Performance Optimization: In the implementation phase, generative AI tools assisted in building parallelized loops and optimizing code execution, which significantly improved runtime—particularly for running the synthetic dataset sweeps that produced Figure~\ref{fig:phase_space}.
\end{itemize}
All scientific decisions, algorithmic designs, and data interpretations were made by me. Generative AI tools were used as productivity aids and companions for debugging, refactoring, and accelerating computation, but not as sources of original scientific ideas or conclusions.