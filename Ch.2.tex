\section{from light to electron}
Human have long history of harnessing light and creating optical microscopes that allow us to look closer. The resolution of the microscope is physically limited by the wavelength of the light, as framed by Abbe's diffraction limit:
\begin{equation}
	d = \frac{\lambda}{2\cdot NA}.
\end{equation}
This states that the smallest distance $d$ at which two points can be distinguished as separate is approximately half of the wavelength of light being used, modified by the numerical aperture $NA$ of the objective lens. This means for a visible light of $\lambda \approx 500nm$, the best possible resolution is around $200$ to $250nm$. By developing light sources with shorter wavelength, scientists were able to push the optical boundary. With the development of UV microscopy in the 1920s, the spatial resolution was pushed to approximately $100nm$. 

People start to look for something with shorter wavelength to resolve finer structures. According to the matter wave theory proposed by Louis de Broglie in 1924, matter exhibits wave-like behavior with a wavelength $\lambda$ inversely proportional to its momentum $p$: 
\begin{equation}
	\lambda = \frac{h}{p},
\end{equation}
where $p$ is the Planck constant. 

This inspired people to look into electrons, whose wavelength is orders of magnitude smaller than that of light, and whose momentum is easier to manipulate through electric field. Ernest Ruska made the most important foundational contributions to electron optics and designed the first electron microscope in 1930s. Borrowing the same set up in a conventional microscope, Ruska designed a transmission electron microscope(TEM), by letting the electron beam piercing through a thin section of the object. The invention of TEM brought the resolution to about $10nm$, achieving the best resolution at that time. 

Scanning tunneling microscope(STM) marks another attempt to manipulate electrons for high-resolution imaging. Yet, unlike traditional transmission microscopy, it utilizes a complete different mechanism called quantum tunneling. In 1981, IBM physicist Gred Binnig and Heinrich Rohrer introduced the first \ac{STM}; by bringing a metallic, atomically sharp tip to the surface of a 



























 
The scanning tunneling microscope(STM), introduced by 1981 by IBM physicists Gred Binnig and Heinrich Rohrer,

\section{Principle of STM and Techniques}


topo mode, STS, grid map
\section{Theory of STM}
\section{STM System in this Study}
%note: talk about the sample plate. 
\footnote{Supplementary: Major Fix of Createc Documentation}
%\section{Remarks on electron scattering under STM}

%\section{STM and transport: 2D and 3D}
%From 2D to 3D, what can we say from STM and what we can not. 1) defect density-> cleaving, scanning etc induced impurities. 2D surface defect number may not represent the bulk-> more 3D materials. 2) defect scattering -> All defect scattering in STM will reflect to the bulk. 
