\chapter{Conclusion and outlooks}
In this thesis, we introduced novel approaches to address challenges in resolving both defect density and scattering features at the defect-specific level. First, we established defect statistics with multinomial distributions; by treating each lattice site as its own incidence of event, we were able to combine many local topographic images and analyze them under the same standard and increase the effective sample size. This way, we increased the accuracy of the 



\section{Summary}

\section{A practical guide to perform MC-SBD on real data}
This section summarizes key insights from the chapter and offers a set of practical guidelines to improve the success rate of applying the algorithm to experimental data.

Although we do not elaborate in detail on best practices for scanning tunneling spectroscopy (STS), it is important to emphasize that the algorithm’s success depends critically on acquiring high-quality grid maps. This involves shaping a stable tip, minimizing vibrational and electronic noise, and maintaining optimal system tuning throughout the measurement. These foundational practices should always be followed. In addition, particular attention must be paid to avoiding structured noise, which can severely degrade algorithm performance.

For optimal results, the grid map should ideally cover a region where the number of occurrences for each defect type of interest exceeds a threshold value, denoted as $N^{\text{critical}}$. Because defect types often vary significantly in their densities, it may be challenging to identify regions where all defect types are sufficiently represented. In such cases, it is advisable to focus on the defect types of greatest interest and select regions where these specific types surpass their respective $N^{\text{critical}}$.

The value of $N^{\text{critical}}$ depends on the signal-to-noise ratio (SNR) of the measurement, which can be estimated by comparing the contrast strength of the QPI pattern around a defect relative to the pristine background. The estimated $N^{\text{critical}}$ based on the SNR is given in the Figure. \ref{fig:KS_vs_N}, which was established in the synthetic data environment. However, due to additional complexities in experimental data, $N^{\text{critical}}$ is shown in Figure. \ref{fig:KS_vs_N} is often underestimated. To mitigate this, it is beneficial to include a larger number of defects by either (i) identifying regions with higher local defect density, or (ii) increasing the physical coverage of the grid. The latter option may be constrained by cryogenic holding time; one possible trade-off is to reduce the real-space resolution. However, as shown in Figures \ref{fig:phase_space} and \ref{fig:phase_spaceN=3}, excessively high local defect density (e.g., $\geq$ 0.1 defects per lattice site) leads to reconstruction failure, so regions with defect clustering should be avoided.

The algorithm allows users to specify the number of kernels and their sizes to deconvolve. The number of kernels should generally match the number of distinct QPI patterns corresponding to the defect types of interest; note that in the case of the same defect with different in-plane rotations, they are considered to have distinct QPI patterns, and we should use multiple kernels. Unrelated defect types can be ignored, as demonstrated in our analyses of LiFeAs (Figure \ref{fig:LiFeAs}) and ZrSiTe (Figure \ref{fig:ZrSiTe1}), where we used two kernels. However, if unwanted defects exhibit strong contrast, they can interfere with the reconstruction. In such cases, we recommend suppressing their signal intensity using truncated Gaussian masks during the preprocessing stage. When choosing kernel sizes, there are two considerations. First, recall the larger kernel sizes will increase the degeneracy space and poses a harder problem for the algorithm, as we discussed in Section 5.2.1, and shown in \ref{fig:ch6_t&s}; and the smallest kernel size should be big enough to cover the whole distinct QPI pattern in the observation $Y$, with the kernel edge being where the QPI signal amplitude meets the noise level. Second, due to the denoising capacity of the algorithm, the output kernel has a lower noise level and could uncover QPI features further from the defect center that was originally inaccessible from the observation $Y$; this sets the upper bound of the kernel size, with which we could retrieve more information in real space and better resolution in momentum space. The user is therefore recommended to start from the smallest and incrementally increase the kernel size until the algorithm fails to produce high-quality outputs. In terms of the kernel shape, while all the kernels used in this chapter are square, the algorithm allows for rectangular kernels, which could be useful for scattering features with spatial anisotropy.

As described earlier, the algorithm is typically run first on a single energy slice (“slice run”), followed by a full energy stack (“block run”) once success has been achieved on the slice. The block run benefits from the output activation maps of the slice run, which can be used as an initial guess. Since defect activations are assumed constant across energy, this initialization is often close to the global minimum of the data fidelity term and significantly improves the block run’s success rate.

Finally, note that the sparsity regularization parameter $\lambda_i^{\text{block}}$ for each kernel type $i$ should be scaled relative to the number of energy slices. Specifically, if the slice run used a value of $\lambda_i^{\text{slice}}$, then the block run with $N_{slices}$ slices should use:
\begin{equation}
	\lambda_i^{\text{block}} = \sqrt{N_{slices}}\lambda_i^{\text{slice}}
\end{equation}
A detailed discussion of the sparsity regularizar $\lambda$ can be reviewed in Section 5.3.2. 

\section{Summary and outlooks}
In this chapter, we presented a comprehensive validation and application of the MC-SBD-STM algorithm, showcasing its ability to demix, deconvolve, denoise, and ultimately extract defect-specific quasiparticle interference (QPI) patterns from both synthetic and experimental scanning tunneling microscopy (STM) data.

We first developed a simulation framework that addressed key limitations in prior synthetic QPI datasets. By grounding kernel truncation in signal-to-noise ratio (SNR) and enforcing lattice-consistent defect placements, we generated realistic test cases for benchmarking. To evaluate algorithm performance, we introduced four metrics—Kernel Similarity (KS), Activation Similarity (AS), Observation Fidelity (OF), and Demixing Score (DS). These enabled a systematic assessment of reconstruction quality across a broad parameter space. Our benchmarks revealed two key dependencies: kernel quality improves with the number of defect occurrences (due to denoising through averaging), while activation fidelity breaks down at high defect densities (due to increased interference). These observations led to the definition of a critical occurrence threshold, $N^{\text{critical}}$, providing a concrete guideline for experimental design. 

We then turned to the application of \ac{MC-SBD} to real STM datasets on four materials: Ag(111), LiFeAs, ZrSiTe, and PtSn\textsubscript{4}. To bridge the gap between simulation and experiment, we developed a preprocessing pipeline to mitigate real-world complexities, including streak noise, periodic environmental noise, and high-contrast defect centers. This standardization step was crucial for algorithm success on experimental data. The algorithm achieved full success on the Ag(111) and ZrSiTe datasets, producing clean, defect-resolved QPI patterns across energies. A partial success was observed in LiFeAs, where the dominant defect types were correctly identified and reconstructed, although the presence of other types reduced the overall observation fidelity. The run on PtSn\textsubscript{4} failed, due to insufficient defect statistics and unresolved experimental artifacts, reaffirming the practical importance of the guidelines established from synthetic benchmarks.

On the note of future studies, our algorithm can be leveraged to explore exotic scattering processes in emerging quantum materials. For example, in ZrSiTe – a nodal-line semimetal where conventional FT-QPI is dominated by a topological drumhead surface state, obscuring the “floating band” scattering observed in its sister compounds \cite{stuartQuasiparticleInterferenceObservation2022}\cite{butlerQuasiparticleInterferenceZrSiS2017}. One of the potential cause is that only one or a few types of defects couples to the "floating band", making its contribution in the FT-QPI obscure. Thus, having access to defect-specific scattering features is crucial to reveal these missing QPI features in momentum space.

On the other hand, by resolving interference from individual defects directly in real space, our approach preserves phase information that is lost in traditional Fourier-based analysis. This opens the door to phase-sensitive studies of scattering in materials like unconventional superconductors. For instance, one could identify Bogoliubov quasiparticle interference patterns governed by selection rules reflecting sign-reversed order parameters, information previously inaccessible via standard QPI techniques \cite{chiSignInversionSuperconducting2014}.