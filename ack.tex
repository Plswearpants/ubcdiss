%% The following is a directive for TeXShop to indicate the main file
%%!TEX root = diss.tex

\chapter{Acknowledgments}
This six-year journey toward a Ph.D. has been, without a doubt, the period of fastest growth in my life so far. While I may have been the one holding the wheel and steering the boat, the journey would not have been nearly as pleasant or fruitful without the support of the wonderful people around me. I love you all and deeply appreciate your presence.

First and foremost, I want to thank Professor Douglas Bonn, who granted me the opportunity to work with him as a Ph.D. student. As my supervisor, Doug has not only guided my studies and trained me to become an independent researcher, but he has also profoundly influenced me with his unwavering commitment to academic integrity. He showed me that “doing the right thing” is not a slogan but a practice — one that can be backed by strong experimental data, rigorous analysis, and careful interpretation presented with humility and clarity. Because of him, I am more aware of my own confirmation bias when doing physics. Doug’s authenticity extends beyond science. As his last Ph.D. student, I had the privilege to witness his early retirement — a deliberate, conscious decision to end his chapter as a physics professor and begin a new one. This inspired me to take authorship of my own path rather than follow the path of least resistance, and to actively choose to pursue a career in AI safety and Human centric AI.

I wish to thank Dr. Seokhwan Choi and Professor Sarah Burke for introducing me to scanning tunneling microscopy (STM) and helping me to visualize this fascinating atomic world. Seokhwan was my first true friend and direct mentor in the LAIR lab during my early Ph.D. days. Scientifically, he taught me the intricacies of operating an STM and passed on the STMer mentality of patience and calm. Whenever I struggled with the tip and felt frustrated, his words, “Have faith in the tip, and it will get better,” came to mind. Outside the lab, Seokhwan has been like an elder brother, caring deeply for my growth and well-being. Sarah, with her deep insights into STM techniques and the underlying physics, became effectively my co-supervisor, especially in the later years of my Ph.D. She challenged me to address the limitations of conventional QPI analysis when multiple defect types are present — a challenge that ultimately led to the second part of my thesis. Despite my not being her direct student, Sarah treated me as one and supported me to the fullest extent. I am deeply grateful for her generosity and mentorship.

I also want to thank Dr. Jisun Kim and Dr. James Day, the best RAs on the planet. Both offered their guidance to me wholeheartedly, both scientifically and personally. Jisun has been my go-to person for STM questions; her wealth of experience and meticulous attention to detail kept me on my toes and held me accountable in experimental design and execution. And yes, Jisun and I share the “Asian stomach” — we both love hot pot! James, the most connected person at QMI and the soul of this institution, always had helpful input for every one of my quests — whether it was finding a hole puncher or figuring out what to do with the rest of my life. He is such a good active listener and always genuinely open to discussing even the dumbest questions, ah right, no dumb questions!. 

Furthermore, I would like to express my appreciation to Dr. Markus Altthaler. Markus has been both a buddy and a mentor, supporting me through many challenges in the second half of my Ph.D. When the Createc STM broke down and underwent major repairs, his knowledge of mechanical systems — and his warm friendship — were the number-one reason I survived that darkest period. He also played a key role in developing the MC-SBD algorithm, helping to establish the initial architecture and iterating with me through its later stages. Our collaboration yielded excellent results. Beyond science, Markus and I often had heated yet nuanced discussions about the world’s problems, making lab life far more engaging and meaningful.

I want to thank the other members of the LAIR group for their friendship and support. Ashley Warner, my co-worker on the Createc STM when it was still running, not only helped me collect data but also patiently worked with me on machine maintenance. Jiabin Yu, Vanessa King, and Rysa Greenwood —  three “Omi-homies” and my wonderful neighbors in both the STM control room and the office — made daily life so much brighter. We shared countless conversations about science, happiness, and the struggles of graduate school. I deeply cherish the friendships we formed and the laughter we shared. Jorn Bannies, now Dr. Bannies, has been both a peer and a dear friend. I will always treasure our science discussions, house parties, and ski trips to the beautiful Whistler mountains.

I am equally grateful to the members of the Supercon group for their encouragement and camaraderie. Mohammed Oudah, the former postdoc in the group, taught me the importance of asking questions, no matter how naive they might seem. And Tim Branch, our cheerful “lab squatter” and fellow student of Doug’s, was always there to share the highs and lows of graduate life. Thank you for cheering me through some of the hard moments, sharing your thesis-writing wisdom that helped me gain momentum, and even correcting my deadlift posture!

I would also like to acknowledge my collaborators — Dr. Alannah Hallas, Samikshya Sahu, and Nicklas Heinsdorf. Working with you showed me that collaboration can be both productive and genuinely joyful. The paper we cracked together was a tough nut, but I believe we did excellent work. I truly value the time we spent as a team and the friendships we built. Samikshya, I still owe you my Mapo Tofu recipe — and Nicklas, stay safe on all your adventures!

I owe the deepest gratitude to my parents, Jianxun Chen and Guiyun Wang. They raised me with abundant love and everything I needed. Growing up without a sense of lack gave me the courage to pursue my dreams. Most importantly, they have always believed in me — even when I lost faith in myself — and that unwavering trust has been my backbone through every challenge.

Finally, I would like to thank my partner, Jingxin Lei, the person brave enough to accept me completely, just as I am, flaws and all. You have shown me that it is truly possible to step into another’s shoes and that pure altruism does exist. Your care and love have given me the space to rest and grow, and have rekindled my hope in this fractured world and the future of humanity. 
