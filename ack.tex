%% The following is a directive for TeXShop to indicate the main file
%%!TEX root = diss.tex

\chapter{Acknowledgments}

This six-year journey to a Ph.D. is no doubt the fastest-growing period of my life so far. While I am the one who took the wheel and sailed the boat, the journey will never be as pleasant and fruitful without the support of the wonderful people around me. I love you all and truly appreciate your presence. 

First, I want to thank Professor Douglas Bonn, who granted me the opportunity to work with him as a PhD student. As my supervisor, Doug has not only guided my study and trained me to be an independent researcher, but also influenced me greatly with his high level of academic integrity. He showed me that doing the right thing is not a slogan, but can be backed with strong experimental data, rigorous analysis, and an interpretation/presentation in a modest and unambiguous fashion.  Because of him, I am more conscious about my potential confirmation bias when practicing physics. Doug’s authenticity go beyond science. As his last PhD student, I have the chance to witness Doug’s early retirement, a deliberate and conscious choice to end his legacy as a physics professor and to start a new life chapter. This encouraged me to take my own authorship rather than choosing the path of the least resistance and make the active decision to pursue my career in AI safety. 

I wish to thank Dr. Seokhwan Choi and Professor Sarah Burke for introducing me to scanning tunneling microscopy (STM) and helping me visualize this wonderful atomic world. Seokhwan was the first real friend I made and my direct mentor in the LAIR lab during my early days of PhD. Scientifically, he taught me the details of operating an STM and passed me the STMer mentality of patience and calm; every time I struggle with the tip and get frustrated, I think of his words, “Have faith in the tip and it will get better.” In daily life, Seokhwan is like an elder brother, who always cares for my growth and my well-being. Sarah, who possesses deep insights into STM technique and the physics behind it, is effectively my co-supervisor, especially in my last few years of PhD. She introduced me to the challenges of conventional QPI analysis when multiple defect types are presented and encouraged me to seek a solution to that, which effectively led to the second part of my thesis. Despite the fact that I am not her own student, Sarah treated me as her own student and supported me to her best ability, for which I am greatly in debt. 

I also want to thank Dr. Jisun Kim and Dr. James Day, the best RAs on the planet. Both of them have offered their help and advice wholeheartedly to me, both scientifically and in life generally. Jisun has been my go-to when I have questions about STM; her wealth of STM experience and meticulous attention to detail always keep me on my toes and hold me accountable when it comes to experimental design and execution. Also, Jisun and I share the asian stomach, and we all love hot pot! James, aka the most connected person of QMI and the soul of this institution, always has a useful input towards all my quests, ranging from where I can find a hole puncher to what I should do for the rest of my life. He is such a good active listener and always genuinely open to discussing even the dumbest questions, ah right, no dumb questions!. 

Furthermore, I would like to express my appreciation to Dr. Markus Altthaler. Markus has been a buddy and a mentor to me, and has helped and supported me going through many challenges in the second half of my PhD. When the Createc STM was broken and underwent a major repair, his knowledge and experience with mechanical parts, together with the warm friendship, were the number one reason why I survived this darkest time. He also contributed greatly to developing the MC-SBD algorithm by helping establish the initial architecture and iterating with me in the later stages. Our teamwork has yielded great results. Beyond academics, Markus and I often have heated but nuanced conversations on the world problems that we care about, which makes my time in the office much more enjoyable! 

I want to thank the members of the LAIR group for their friendship and support. Ashley Warner, my co-worker on the Createc STM back when it was running, not only helped me collect data but also patiently performed machine maintenance with me. Jiabin Yu, Vanessa King, and Rysa Greenwood — three “Omi-homies” and my wonderful neighbors in both the STM control room and the office — made daily life so much brighter. We shared countless conversations about science, our happiness, and the struggles of graduate school. I truly cherish the friendship we built and the laughter we shared along the way. Jorn Bannies,  now Dr. Bannies, has been both a great friend and peer. I will always treasure our science discussions, house parties, and ski trips to the beautiful Whistler mountains. 
I am equally grateful to the Supercon group for their encouragement and camaraderie. Mohammed Oudah, the former postdoc in the group, taught me the importance of asking questions, no matter how naive they might seem. And Tim Branch, our cheerful “lab squatter” and fellow student of Doug’s, was always there to share the highs and lows of graduate life. Thank you for cheering me on during the tough moments, sharing your thesis-writing experience that helped me build momentum, and even correcting my deadlift posture! 

