\chapter{Quasiparticle Interference (QPI)}
%todo: to write the abstract of this chapter after finishing this chapter
\section{Introduction to Quasiparticle Interference measurement}

\subsection{what is \ac{QPI}}
Theoretical concept of QPI: QPI as scattering probe, QPI and band structure
The goal of QPI is to gain knowledge of the band structure of a material, but since the QPI pattern is the signature for scattering, we need scattering in the material, and also, we need to focus our area of view around these scattering events, and those events occur where the defects are reside. 


\subsection{\ac{QPI} simulation-toy model}
We simulate the electron density modulation from a point defect placed in a squared lattice modeled by Tight-binding model.
\subsubsection{the process}
\subsubsection{the result}
\subsection{\ac{QPI} simulation in physical systems}

\section{QPI measurements}
How do we measure QPI pattern with STM, the process and analysis we do 
\subsubsection{QPI quality}
Factors determining quality of a QPI measurement, and what is an ideal quality. 
1. high resolution in q-space <-> range in real-space
2. high range in q-space <-> resolution in real-space
3. low noise level-> machine's intrinsic noise level, voltage sweeps and averaging.
4. no phase noise 
\subsubsection{challenges to obtain optimal QPI measurements}
