Order and disorder are ontological states of being—realities that exist prior to our attempts to interpret them. Nature does not distinguish between order and disorder; it simply unfolds. It is we, the observers, who draw the line—tracing the boundary between what we understand and what we have yet to comprehend. In this light, the history of science becomes a record of those shifting lines: what once appeared as chaos or randomness gradually revealed itself to be structured and patterned; and from the known structures and patterns, new disorders arose.

Take the stars. To ancient eyes, a few bright points wandered unpredictably across the night sky, sometimes even reversing their path. These “planētai”, from the Greek for 'wanderers', were marked as celestial anomalies. Through over a thousand years of effort to understand the system through complex models with many corrections, planetary motions remained disordered, as they defied the elegant circles philosophers believed to rule the heavens. It was not until the late 1500s, when Tycho Brahe recorded the positions of planets with the best precision of his time, that people began to question the validity of the circular models, as the data clearly deviated from them. And in 1609, Johannes Kepler redrew the boundary. With his discovery that planetary orbits are ellipses, not circles, the apparent disordered motion shifted into a new kind of order—understood as the projection of elliptical orbits from a geocentric perspective.

A similar story unfolded in 1827, when botanist Robert Brown observed tiny pollen grains dancing erratically in water. The motion seemed disordered, described by Brown as “life-like.” Yet in 1905, Einstein proposed a theory behind the “Brownian Motion” using the then-called molecular-kinetic theory of heat \cite{Einstein1905}, and showed that this jitter was not meaningless, but potentially the statistical consequence of countless, ordered molecular collisions. Just three years later, Jean Perrin experimentally confirmed this theory by looking closer, using a high-power microscope to track particle motion with precision \cite{r.BrownianMovementMolecular1911}. His work not only validated Einstein’s model but also provided strong evidence for the discrete, molecular structure of matter. For this, Perrin was awarded the Nobel Prize in Physics in 1926 \cite{ https://www.nobelprize.org/prizes/physics/1926/perrin/lecture/}.

These historical episodes remind us that what we call "disorder" is often structure we haven't yet resolved—and that the path toward understanding is driven by two fundamental forces. First, the drive to look closer: to improve our ability to observe what was previously invisible, by increasing the accuracy and precision of experimental measurements. Second, the drive to think differently: to reinterpret what we observe, forming new theoretical models that both explain current data and forecast the system’s behavior. These two forces often work in tandem: new perspectives inspire the invention of new tools, and new tools reveal patterns that force a change in perspective.

This crafts a lens that is both optimistic; since the line between order and disorder is not a border that is firmly stamped in nature, every disorder invites investigation with the hope to advance our understanding; and humbling, as each new insights typically exposes further layers of the unknown. It is through this lens that I wish to present the work in this thesis. My research is an attempt to uncover defect-specific information in crystalline materials—systems that, at first glance, are among the most ordered known to us. Defects are often treated as irregularities or noise, but in reality, they carry rich information about local symmetry breaking, scattering behavior, and material response. They are the disorders within an ordered lattice, and by examining them carefully—both by looking closer with experimental tools like scanning tunneling microscopy, and by thinking differently through algorithmic deconvolution methods—we can better resolve the structure embedded within. 

In this chapter, to provide a comprehensive introduction, I will begin with an overview of the history of defect studies, followed by a review of modern understanding and applications of defects. I will then discuss why defect-specific information is needed, and explore the reasons why this information is often missing in current methodologies. Finally, I will close with an overview of this thesis and present the logic behind the work that follows.
