%% The following is a directive for TeXShop to indicate the main file
%%!TEX root = diss.tex

\chapter{Glossary}

%This glossary uses the handy \latexpackage{acroynym} package to automatically maintain the glossary.  It uses the package's \texttt{printonlyused} option to include only those acronyms explicitly referenced in the \LaTeX\ source.  To change how the acronyms are rendered, change the \verb+\acsfont+ definition in \verb+diss.tex+.


% The acronym environment will typeset only those acronyms that were
% *actually used* in the course of the document
\begin{acronym}
\acro{ARPES}{Angle-Resolved Photoemission Spectroscopy}  A technique that measures the energy and momentum of photoemitted electrons to map the electronic band structure.  
\acro{BD}{Blind Deconvolution}  The general problem of recovering both the input signal and convolution kernel from their convolution without prior knowledge.  
\acro{BLGF}{Bare Lattice Green's Function}  The Green’s function of a lattice without impurities or interactions, serving as the reference propagator.  
\acro{BZ}{Brillouin Zone}  The primitive cell of the reciprocal lattice, containing all unique crystal momenta.  
\acro{CEC}{Constant Energy Contour}  The set of momentum points in the Brillouin zone with the same electronic energy.  
\acro{FT}{Fourier Transform}  A mathematical operation that converts a function from real space to reciprocal (frequency or momentum) space.  
\acro{FT-STS}{Fourier-Transform Scanning Tunneling Spectroscopy} The Fourier transform of spatially resolved STS data, revealing scattering wavevectors and electronic structure.  
\acro{JDOS}{Joint Density of States}  The momentum-resolved density of states weighted for all possible pairs of states connected by a given scattering vector.  
\acro{LDOS}{Local Density of State} The spatially resolved density of electronic states at a given energy.  
\acro{MC-SBD}{Multi-Channel Sparse Blind Deconvolution} An algorithm that jointly deconvolves multiple channels of data under a shared sparse kernel assumption.  
\acro{QPI}{Quasiparticle Interference} Standing-wave patterns formed by elastically scattered quasiparticles, observable with STM/STS.  
\acro{RRR}{Residual Resistivity Ratio} The ratio of resistivity at room temperature to that at low temperature, used as a measure of crystal purity.  
\acro{RTRM}{Riemannian Trust Region Method} An optimization algorithm on Riemannian manifolds that iteratively refines solutions using second-order local models.  
\acro{SBD}{Sparse Blind Deconvolution} A signal-processing method that recovers both a sparse activation map and an unknown convolutional kernel.  
\acro{STM}{Scanning Tunneling Microscope} An instrument that maps surfaces with atomic resolution by measuring quantum tunneling current between a sharp tip and the sample.  
\acro{STS}{Scanning Tunneling Spectroscopy} A technique using STM to measure differential conductance as a function of bias, probing the Local density of states.  
\acro{TB}{Tight-Binding} A lattice model describing electrons as hopping between localized orbitals, often used to compute band structures.  
\acro{UHV}{Ultra-High Vacuum} An environment with pressures below \(10^{-9}\,\mathrm{mbar}\).

\end{acronym}

% You can also use \newacro{}{} to only define acronyms
% but without explictly creating a glossary
% 
% \newacro{ANOVA}[ANOVA]{Analysis of Variance\acroextra{, a set of
%   statistical techniques to identify sources of variability between groups.}}
% \newacro{API}[API]{application programming interface}
% \newacro{GOMS}[GOMS]{Goals, Operators, Methods, and Selection\acroextra{,
%   a framework for usability analysis.}}
% \newacro{TLX}[TLX]{Task Load Index\acroextra{, an instrument for gauging
%   the subjective mental workload experienced by a human in performing
%   a task.}}
% \newacro{UI}[UI]{user interface}
% \newacro{UML}[UML]{Unified Modelling Language}
% \newacro{W3C}[W3C]{World Wide Web Consortium}
% \newacro{XML}[XML]{Extensible Markup Language}
