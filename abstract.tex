%% The following is a directive for TeXShop to indicate the main file
%%!TEX root = diss.tex

\chapter{Abstract}

Point defects in quantum materials, once regarded as imperfections, are now recognized as central players in shaping electronic properties. They can scatter quasiparticles, modify local order, and in some cases host entirely new states of matter. Scanning tunneling microscopy provides a uniquely powerful window into these effects, combining atomic-scale spatial resolution with spectroscopic access to electronic structure. Yet in practice, the contributions of different defect types are typically entangled, obscuring their individual roles.

In this thesis, I develop two complementary approaches to address this problem. First, I establish a statistical framework to estimate the global densities of each type of defect directly from scanning tunneling microscopy topographs. Applied to the ultra-pure semimetal PtSn\textsubscript{4}, this approach demonstrates how nanoscale imaging can yield quantitative information about the populations of distinct defect species across macroscopic crystals. Second, I introduce a new analysis method—multi-channel sparse blind deconvolution tailored for scanning tunneling data—designed to disentangle overlapping quasiparticle interference patterns into defect-specific scattering fingerprints. Benchmarks on synthetic datasets identified the conditions required for reliable reconstruction, and application to experimental data produced defect-resolved interference patterns in several systems. Full success was achieved on Ag(111) and ZrSiTe, partial success on LiFeAs, and failure on PtSn\textsubscript{4}, where the limitations of the method matched expectations from simulations. In ZrSiTe in particular, the defect-resolved interference pattern for the Zr\textsubscript{2} defect revealed clear evidence of floating-band scattering features, which had previously been reported as absent.

\ifgpscopy
  This document was typeset in \texttt{gpscopy} mode.
\else
  This document was typeset in non-\texttt{gpscopy} mode.
\fi

% Consider placing version information if you circulate multiple drafts
%\vfill
%\begin{center}
%\begin{sf}
%\fbox{Revision: \today}
%\end{sf}
%\end{center}
